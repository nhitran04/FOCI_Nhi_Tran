% Introduction

The imposition of online speech exposes vulnerabilities among citizens in nation-state regimes by restricting their access to new knowledge and impeding their ability to form unique perspectives for the betterment of society. Censorship evasion is a constantly evolving field that is highly active in computer science research. Geneva, a genetic evasion algorithm, surpasses censors by manipulating the packet stream on one end of a connection, whether that would be on the server or client side. The algorithm is composed of four packet-level actions that operate together as censorship evasion strategies and allows us to manipulate packets to confuse the regime’s censors without destroying the connection. Rather than manually creating strategies, Geneva actually learns how to evade censors by finding bugs in them. This provides us with an insight into the various techniques nation-state regimes use to censor people online.

Previous implementations of censorship evasion tools required active client participation since it required users to install their software. These anti-censorship techniques are counterproductive, however, because they themselves can be censored and can put users at risk of exposure. Geneva avoids these issues with the help of server-side evasion, eliminating the need for client participation. Rather than having the user install any software, the server can allow them to access the information by redirecting the traffic and eliminating censorship for them.

A core limitation of previous approaches to censorship evasion (Geneva, SymTCP, Alembic, and others) is that they all required client-side instrumentation for their training phase. Even when packet manipulation strategies are deployed exclusively at the server, each tool still requires the control of a client to issue requests through a censor. This poses a significant limitation: it is challenging for researchers to get access to machines in every network of interest. 

The aim of our research is to obtain global penetration into more networks than we could have access to on our own. Our insight is that we can expand participation in online censorship evasion without the need for users to download the Geneva software. To accomplish this, we propose a webpage that users can visit to perform the client-side of server-side training for new Geneva strategies.

